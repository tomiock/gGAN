Desde hace más de un año, me he dedicado a estudiar computación cuántica durante mi tiempo libre. Buscaba investigar un campo relacionado con la mecánica cuántica, pero sin que sea muy complicado, que se pueda entender a un nivel teórico y que me entusiasme.

La Computación Cuántica (concretamente Teoría de la Información Cuántica) encaja perfectamente con esos criterios. Es más sencilla que la mecánica cuántica debido a que no está basada en cálculo o ecuaciones diferenciales, se basa en la álgebra lineal. Siempre se emplean valores discretos, vectores y matrices. Además si se trabaja a un nivel teórico sencillo no se tienen en consideración las interpretaciones físicas, lo cual simplifica mucho las cosas. Cuanto más me adentraba, más ganas tenía de seguir. 

Mi parte favorita de este campo es el Quantum Machine Learning que consiste en diseñar y aplicar conceptos de Machine Learning  a los ordenadores cuánticos, como por ejemplo implementar cuánticamente las famosas Redes Neuronales, que estan detrás de la mayoria de inteligencias artificiales que vemos hoy en día.

QML es un campo de investigación joven y en crecimiento debido a que sus algoritmos son ideales para implementarlos con los ordenadores cuánticos actuales, los cuales no son muy potentes. Ejemplos de estas implementaciones serían algoritmos de generación y clasificación de dígitos, clasificación de datos no etiquetados, control de operaciones cuánticas, análisis de datos de aceleradores de partículas, etc.

De entre todos los tipos de algoritmos me he centrado en las Redes Neuronales Cuánticas, análogas cuánticas de las Redes Neuronales tan utilizadas hoy en día para hacer gran variedad de tareas. Me he interesado particularmente en ellas debido a que tenía experiencia en el pasado con las RR clásicas y había visto que existen frameworks de software para trabajar con ellas como TensorFlow Quantum que me podían ayudar. 

Para adentrarme en el campo de QML, he tenido que adquirir conocimientos en álgebra lineal, cálculo y física. Dentro de QML en concreto me he dedicado a leer papers que me interesan y en un par de ocasiones intentar implementar los algoritmos detallados en esos papers. Puede parecer algo imposible en principio debido a que no tengo acceso directo a un ordenador cuántico,  no obstante estos no son necesarios debido a que las operaciones cuánticas pueden ser simuladas en un ordenador corriente de escritorio (con ciertas limitaciones). Pero en reaque puedo tener acceso a ordenadores cuánticos ya que IBM permite acceder a los suyos mediante IBM Quantum Experience, aunque nunca he dado uso de ello debido a que no lo veía necesario.

En este trabajo de investigación me he propuesto implementar mediante código uno de los algoritmos que he visto en un paper, una Red Adversaria Generativa Cuàntica (GAN, en inglés) que genera dígitos binarios (unos y ceros) a partir de un circuito cuántico. Como objetivo tengo verificar una sugerencia que hacen los autores del paper: implementar una función no-lineal en una parte del algoritmo que podría mejorar el rendimiento de este. Mi hipótesis al igual que los autores (aunque ellos lo comentan muy brevemente) es que el algoritmo va a reducir ligeramente el número de interacciones que son necesarias para llegar a su punto óptimo. Es decir, el modelo con la función no-lineal va a necesitar menos operaciones que lo entren conseguir los mismos resultados que el modelo sin la función.

%Marco teórico
%Machine Learning
%Una de las habilidades más importantes que tenemos como especie es el aprendizaje. Somos excelentes a la hora de reconocer patrones y aprender de ellos, una habilidad que las computadoras no tienen de por sí. Al menos hasta finales de 1960, cuando se creó el primer programa de aprendizaje automático. Consistía en un pequeño algoritmo que aprendió a jugar a las damas. Básicamente se entrenaba a sí mismo analizando sus juegos anteriores y manuales para aprender a jugar a las damas. [Reference] Se convirtió en el primer ejemplo de un programa al que proporcionarle datos, cambia las operaciones de las cuales está compuesto y de esa mejora su rendimiento.

%Este tipo de programas forman parte del aprendizaje automático, un subcampo de las Ciencias de la Computación. Durante la última década ha adquirido una importancia considerable debido al Aprendizaje Profundo o Deep Learning, un subcampo del AP en el cual se desarrollan modelos de inteligencia artificial muy poderosos. [Reference] Este tipos de modelos están detrás de la muy amplia mayoría de programas que popularmente se conocen como inteligencias artificiales, como ejemplo tenemos todos los asistentes virtuales que se activan mediante comandos de voz, los traductores online más populares, algoritmos de recomendación de videos o productos y generadores de texto, video o imágenes. [Reference] 

%Se considera que el Aprendizaje Automático es una parte vital del mundo de la computación debido a la potencia de los modelos de Aprendizaje Profundo y las valiosas aplicaciones que tienen estos en el mundo real.
%Computación Cuántica
%Otro de los subcampos más prometedores y emocionantes de las Ciencias de la Computación es la Computación Cuántica. Este se basa alrededor de la Teoría de Información Cuántica, en la cual se estipula que la información en vez de presentarse en bits se presenta en qubits. 

%Los qubits se pueden entender como bits de información (una cadena de zeros y unos) pero que presentan ciertas propiedades cuánticas como superposición y entrelazamiento cuánticos. Estas propiedades permiten diseñar algoritmos mucho más eficientes o computacionalmente más baratos. Los mejores ejemplos son el algoritmos de Shor para la factorización en números primos y el algoritmo de Grover para búsqueda en una secuencia de datos no ordenada. Estos algoritmos resuelven problemas los cuales ya se pueden resolver sin la computación cuántica, pero estos lo hacen de una manera muchísimo más rápida. Como ejemplo tenemos al de Shor: la factorización en números primos de números grandes con algoritmos clásicos (i.e. no cuánticos) se considera intractable. Se tardaría tanto tiempo en factorizar que aunque se use toda la potencia computacional de la humanidad no tendría sentido hacerlo. Es por esto que prácticamente toda la criptografía actual se basa en multiplicar y factorizar números en números primos, debido a que es una one way operation, se pueden multiplicar por números primos fácilmente, pero si no los conoces es imposible hacer la operación inversa; la factorización [Reference].

%No obstante estos algoritmos cuánticos no son útiles debido a que los ordenadores cuánticos, en los cuales se deben de ejecutar están en un estado demasiado maduros. No llegan a ser lo suficientemente potentes para poder ejecutarlos y llegar a conseguir con estos una ventaja sobre los ordenadores cuánticos clásicos. Los ordenadores cuánticos sufren de lo que se denomina decoherencia cuántica, los estados cuánticos que representan la información dejan de tener propiedades cuánticas debido al colapso de la función de onda y si estaban en el proceso de ejecutar una operación, esta dejará de ser válida. Además las operaciones que se aplican a los estados no son precisas, tienen un error asociado a ellas. Dependiendo del ordenador, este error puede ser mayor o menor, debido a esto no se pueden ejecutar algoritmos que requieran de gran precisión y se tienen que recurrir a error correction codes para poder ejecutar la mayor parte de las operaciones [References]. 


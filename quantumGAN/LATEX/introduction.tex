Aquest treball de recerca té l'objectiu de desenvolupar un model de intel·ligència artificial quàntica. Concretament és tracta d'una xarxa neuronal generativa adversària, que té l'objectiu de generar imatges \cite{QGAN_exp, GAN2014}. En aquest model un circuit quàntic és l'encarregat de generar les imatges.

Com a pregunta a investigar m'he proposat verificar la utilitat d'una funció no lineal que implementen els autors en els circuits quàntics. Segons els autors aquesta funció millora el rendiment del model, és a dir, que les imatges són generades amb una major eficiència \cite{QGAN_exp}.

Des de fa més de dos anys, m'he dedicat a estudiar computació quàntica durant el meu temps lliure. Buscava investigar un camp relacionat amb la mecànica quàntica, però sense que sigui molt complicat; un camp que pugui entendre a un nivell teòric i que m'entusiasmi.

La Computació Quàntica encaixa perfectament amb aquests criteris. És més senzilla que la mecànica quàntica pel fet que no està basada en càlcul o equacions diferencials. En canvi, es basa en l'àlgebra lineal; utilitzant valors discrets, vectors i matrius. A més a més, si es treballa a un nivell teòric senzill, no s'han de tenir en consideració les interpretacions físiques, aquest factor simplifica molt el procés.

La meva part favorita d'aquest camp és el \textit{Quantum Machine Learning} (QML) que consisteix a dissenyar i aplicar conceptes de \textit{Machine Learning} als ordinadors quàntics, com per exemple implementar en circuits quàntics les famoses xarxes neuronals, que estan darrere de la majoria d'intel·ligències artificials que veiem avui dia \cite{schuld:2014}.

QML és un camp de recerca jove i en creixement, ja que els algoritmes d'aquest camp són ideals per a implementar-los amb els ordinadors quàntics actuals, els quals no són molt potents.

D'entre tots els tipus d'algoritmes que existeixen dintre de QML m'he centrat en les xarxes neuronals quàntiques, anàlogues quàntiques de les xarxes neuronals tan utilitzades avui dia per a fer una gran varietat de tasques. M'he interessat particularment en elles pel fet que tenia experiència en el passat amb les xarxes neuronals clàssiques, i havia vist que existeixen \textit{frameworks} de programació per a treballar amb elles com \textit{TensorFlow Quantum} \cite{tfq} que em podien ajudar.

Per a endinsar-me en el camp de QML, vaig haver d'adquirir coneixements en àlgebra lineal, càlcul i física. Un cop apressos aquests conceptes em vaig dedicar a llegir papers que em cridaven l'atenció. Fins i tot, en un parell d'ocasions vaig intentar implementar alguns algoritmes en \textit{Python}. Tenia molt clar que faria el treball de recerca sobre aquest camp. Durant aquest temps vaig llegir l'article sobre xarxes generatives adversàries (GAN en àngles) quàntiques que em va convèncer per fer un experiment sobre aquests algoritmes \cite{QGAN_exp}.





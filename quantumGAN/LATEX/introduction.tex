Des de fa més de dos anys, m'he dedicat a estudiar computació quàntica durant el meu temps lliure. Buscava investigar un camp relacionat amb la mecànica quàntica, però sense que sigui molt complicat, un camp que pugui entendre a un nivell teòric i que m'entusiasmi.

La Computació Quàntica encaixa perfectament amb aquests criteris. És més senzilla que la mecànica quàntica pel fet que no està basada en càlcul o equacions diferencials; en canvi, es basa en l'àlgebra lineal, utilitzant valors discrets, vectors i matrius. A més a més, si es treballa a un nivell teòric senzill, no s'han de tenir en consideració les interpretacions físiques, aquest factor simplifica molt les coses.

La meva part favorita d'aquest camp és el \textit{Quantum Machine Learning} (QML) que consisteix a dissenyar i aplicar conceptes de \textit{Machine Learning} als ordinadors quàntics, com per exemple implementar en circuits quàntics les famoses xarxes neuronals, que estan darrere de la majoria d'intel·ligències artificials que veiem avui dia \cite{schuld:2014}.

QML és un camp de recerca jove i en creixement, ja que els algoritmes d'aquest camp són ideals per a implementar-los amb els ordinadors quàntics actuals, els quals no són molt potents.

D'entre tots els tipus d'algorismes que existeixen dintre de QML m'he centrat en les xarxes neuronals quàntiques, anàlogues quàntiques de les xarxes neuronals tan utilitzades avui dia per a fer una gran varietat de tasques. M'he interessat particularment en elles pel fet que tenia experiència en el passat amb les xarxes neuronals clàssiques i havia vist que existeixen \textit{frameworks} de programació per a treballar amb elles com \textit{TensorFlow Quantum} \cite{tfq} que em podien ajudar.

Per a endinsar-me en el camp de QML, vaig haver d'adquirir coneixements en àlgebra lineal, càlcul i física. Una vegada havia vist aquests conceptes em vaig dedicar a llegir articles que m'interessen i en un parell d'ocasions fins i tot vaig intentar implementar aquests algoritmes en \textit{Python}. Pot semblar una cosa impossible, ja que, no tinc accés directe a un ordinador quàntic, no obstant això, aquests no són necessaris perquè les operacions quàntiques poden ser simulades en un ordinador corrent d'escriptori. Però puc tenir accés a ordinadors quàntics, ja que IBM permet accedir als seus mitjançant \textit{IBM Quantum Experience} \cite{IBM_Q}, encara que mai he donat ús d'això degut a que no ho veia necessari.

En aquest treball de recerca m'he proposat implementar mitjançant codi un dels algorismes que he vist en un article, una Xarxa Adversària Generativa Cuàntica (GAN, en anglès) \cite{GAN2014} que genera imatges a partir d'un circuit quàntic \cite{QGAN_exp}.

Com a pregunta a investigar m'he proposat verificar la utilitat d'una funció no lineal que implementen els autors en els circuits quàntics que generen les imatges. Segons els autors aquesta funció millora el rendiment del model, és a dir, que les imatges són generades amb una major eficiència \cite{QGAN_exp}.



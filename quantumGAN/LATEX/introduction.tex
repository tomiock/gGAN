Desde hace más de un año, me he dedicado a estudiar computación cuántica durante mi tiempo libre. Buscaba investigar un campo relacionado con la mecánica cuántica, pero sin que sea muy complicado, que se pueda entender a un nivel teórico y que me entusiasme.

La Computación Cuántica encaja perfectamente con esos criterios. Es más sencilla que la mecánica cuántica debido a que no está basada en cálculo o ecuaciones diferenciales, se basa en la álgebra lineal, utilizando valores discretos, vectores y matrices. Además si se trabaja a un nivel teórico sencillo, no se tienen en consideración las interpretaciones físicas, lo cual simplifica mucho las cosas. Cuanto más me adentraba, más ganas tenía de seguir. 

Mi parte favorita de este campo es el Quantum Machine Learning que consiste en diseñar y aplicar conceptos de Machine Learning  a los ordenadores cuánticos, como por ejemplo implementar cuánticamente las famosas Redes Neuronales, que están detrás de la mayoría de inteligencias artificiales que vemos hoy en día \cite{schuld:2014}.

QML es un campo de investigación joven y en crecimiento debido a que sus algoritmos son ideales para implementarlos con los ordenadores cuánticos actuales, los cuales no son muy potentes. Ejemplos de estas implementaciones serían [insertar aplicaciones aquí], etc.

De entre todos los tipos de algoritmos me he centrado en las Redes Neuronales Cuánticas, análogas cuánticas de las Redes Neuronales tan utilizadas hoy en día para hacer gran variedad de tareas. Me he interesado particularmente en ellas debido a que tenía experiencia en el pasado con las RNs clásicas y había visto que existen frameworks de software para trabajar con ellas como TensorFlow Quantum \cite{tfq} que me podían ayudar. 

Para adentrarme en el campo de QML, he tenido que adquirir conocimientos en álgebra lineal, cálculo y física. Dentro de QML en concreto me he dedicado a leer papers que me interesan y en un par de ocasiones intentar implementar los algoritmos detallados en esos papers. Puede parecer algo imposible en principio debido a que no tengo acceso directo a un ordenador cuántico,  no obstante estos no son necesarios debido a que las operaciones cuánticas pueden ser simuladas en un ordenador corriente de escritorio (con ciertas limitaciones). Pero puedo tener acceso a ordenadores cuánticos ya que IBM permite acceder a los suyos mediante IBM Quantum Experience \cite{IBM_Q}, aunque nunca he dado uso de ello debido a que no lo veía necesario.

En este trabajo de investigación me he propuesto implementar mediante código uno de los algoritmos que he visto en un paper, una Red Adversaria Generativa Cuàntica (GAN, en inglés) \cite{GAN2014} que genera imágenes a partir de un circuito cuántico \cite{QGAN_exp}. Como objetivo tengo verificar una sugerencia que hacen los autores del articulo: implementar una función no-lineal en una parte del algoritmo que podría mejorar el rendimiento de este. Mi hipótesis al igual que los autores (aunque ellos lo comentan muy brevemente) es que el algoritmo va a reducir ligeramente el número de interacciones que son necesarias para llegar a su punto óptimo. Es decir, el modelo con la función no-lineal va a necesitar menos operaciones que lo entren conseguir los mismos resultados que el modelo sin la función.



\chapter{Més àlgebra lineal}
Ja he escrit bastants pàgines sobre àlgebra lineal, però aparentment no eren les suficients perquè estic content amb el treball\footnote{Hi han moltes coses guays i interessants que vull explicar.}, així que aquí hi ha més àlgebra lineal.


\section{Procediment de Gram–Schmidt}\label{gram}
El procediment de Gram-Schmidt és un mètode utilitzat per produir bases per a espais vectorials \cite{QCandQI:GramSchmidt}. Per un espai $V$ amb producte interior de $d$ dimensions amb el set de vectors ${\ket{v_1}, \cdots, \ket{v_d}}$, podem definir una nova base de vectors ortonormals $\{\ket{u}\}$. El primer element d'aquest set és $\ket{u_1} = \ket{v_1}/\norm{\ket{v_1}}$, amb el següent element $\ket{v_{k+1}}$ sent:
$$
\ket{u_{k+1}} = \frac{\ket{v_{k+1}} - \sum_{i=1}^{k} \bra{u_i}\ket{v_{k+1}}\ket{u_i}}{\norm{\ket{v_{k+1}} - \sum_{i=1}^{k} \bra{u_i}\ket{v_{k+1}}\ket{u_i}}}
$$
Per $k$ en el interval $1 \leq k \leq d-1$.

Si seguim per $k$ en $1 \leq k \leq d-1$, obtenim el set de vectors ${\ket{u_1}, \cdots , \ket{u_d}}$ que es una base vàlida per l'espai ortonormal per l'estai $V$. Els vectors creats han de tindre el mateix span\footnote{L'span d'un set de vectors són totes les combinacions lineals possibles amb aquests vectors.} que el dels vectors que originalment eren la base per $V$:
$$
\text{span}(\{\ket{v}\}) = \text{span}(\{\ket{v}\}) = V
$$
Cal notar que l'span de del set base és la definició del espai. En altres paraules, cada vector en $V$ pot ser representat per una combinació del vectors base. 

La prova de que és una base ortonormal és bastant simple:
Podem veure immediatament que els elements  de $\{ \ket{u} \}$ són vectors unitaris perquè estan normalitzats (els vectors $\ket{v_{k+1}} - \sum_{i=1}^{k} \bra{u_i}\ket{v_{k+1}}\ket{u_i}$ estan dividits per la seva norma). També podem veure que són ortogonals els uns als altres mirant que el producte interior entre els doni 0:

Per $k=1$:
$$
\ket{u_2} = \frac{\ket{v_2} - \bra{u_1}\ket{v_2}\ket{u_1}}{\norm{\ket{v_2} - \bra{u_1}\ket{v_2}\ket{u_1}}}
$$
Per tant el producte interior amb $\ket{v_1}$ és:
$$
\begin{aligned}
\bra{u_1}\ket{u_2} 
&= \bra{u_1} \left( \frac{\ket{v_2} - \bra{u_1}\ket{v_2}\ket{u_1}}{\norm{\ket{v_2} - \bra{u_1}\ket{v_2}\ket{u_1}}} \right) \\
&= \frac{\bra{u_1}\ket{v_2} - \bra{u_1}\ket{v_2}\bra{u_1}\ket{u_1}}{\norm{\ket{v_2} - \bra{u_1}\ket{v_2}\ket{u_1}}} \\
&= 0
\end{aligned}
$$
Per inducció podem veure que per $j \leq d$, amb $d$ sent la dimensió del espai vectorial:
$$
\begin{aligned}
	\bra{u_j}\ket{u_{n+1}} 
	& = \bra{u_j} \left( \frac{\ket{v_{n+1}} - \sum_{i=1}^{n} \bra{u_i}\ket{v_{n+1}}\ket{u_i}}{\norm{\ket{v_{n+1}} - \sum_{i=1}^{n} \bra{u_i}\ket{v_{n+1}}\ket{u_i}}} \right) \\
	& = \frac{\bra{u_j}\ket{v_{n+1}} - \sum_{i=1}^{n} \bra{u_i}\ket{v_{n+1}} \bra{u_j}\ket{u_i}}{\norm{\ket{v_{n+1}} - \sum_{i=1}^{n} \bra{u_i}\ket{v_{n+1}} \ket{u_i}}} \\
	& = \frac{\bra{u_j}\ket{v_{n+1}} - \sum_{i=1}^{n} \bra{u_i}\ket{v_{n+1}} \delta_{ij}}{\norm{\ket{v_{n+1}} - \sum_{i=1}^{n} \bra{u_i}\ket{v_{n+1}} \ket{u_i}}} \\
	& = \frac{\bra{u_j}\ket{v_{n+1}} - \bra{u_j}\ket{v_{n+1}}}{\norm{\ket{v_{n+1}} - \sum_{i=1}^{n} \bra{u_i}\ket{v_{n+1}} \ket{u_i}}} \\
	& = 0	\\
\end{aligned}
$$
Tot això no és veu molt clar al principi però cal recordar que el producte interior de dos vectors ortonormals és zero, i que el producte interior entre el mateix vector unitari és un.

\section{Curs ràpid de la notació de Dirac}
A la taula següent hi ha un resum de conceptes matemàtics de l'àlgebra lineal importants expressats en la notació de Dirac\footnote{La notació utilitzada per un espai vectorial complex i l'espai dels nombres complex no són de la notació de Dirac estàndard, però les poso per explicar el que signifiquen.} \cite{QCandQI:dirac_notation}

\begin{tabular}{ p{2cm}|p{12cm} }
	\hline
	Notació & Descripció\\
	\hline
	\hline
	$z$ & Nombre complex    \\
	$z^{*}$ & Conjugat complex d'un nombre complex $z$. $(a+ bi)^{*} = (a -bi)$\\
	$\ket{\psi} $ & Vector amb una etiqueta $\psi$. Conegut com \textit{ket}\\
	$\ket{\psi}^T$ & Transposada d'un vector $\ket{\psi}$ \\
	$\ket{\psi}^\dag $ &  Conjugat Hermitià d'un vector. $\ket{\psi}^\dag = (\ket{\psi}^T)^* $\\
	$\bra{\psi} $ & Vector dual a $\ket{\psi}$. $ \ket{\psi} = \bra{\psi}^{\dag}$ i $\bra{\psi} = \ket{\psi}^\dag$. Conegut com \textit{bra}\\
	$ \bra{\varphi}\ket{\psi} $ & Producte interior dels vectors $\bra{\varphi}$ i $\ket{\psi}$ \\
	$ \ket{\varphi}\bra{\psi} $ & Producte exterior del vectors $\bra{\varphi}$ i $\ket{\psi}$ \\
	$ \ket{\psi}\otimes\ket{\varphi}$ & Producte tensorial del vectors $\ket{\varphi}$ i $\ket{\psi}$ \\
	$ 0 $ & Vector zero i operador zero \\
	$ \mathbb{I}_n $ & Matriu identitat de dimensions $n\times n$ \\
	$ \mathbb{C}_n $ & Espai vectorial complex de dimensió $n$ \\
	$ \mathbb{C}_1$ o $\mathbb{C} $ & Espai del nombres complexos \\
	
\end{tabular}
\section{Més on la traça parcial}



\chapter{Quantum Computation vs Quantum Mechanics}
On the introduction I mentioned that one of the reasons for with I started learning and researching quantum computing is because it is easy, on this appendix I am going to present why this is the case with a practical example.

On quantum mechanics the more general way to represent quantum state, like the orbitals of an atom of hydrogen, are wavefunctions, not statevectors. Wavefunctions are extremely useful, however, working with them adds a hole new level of complexity because they are continuous functions that depend on time. Compare them with vectors, which are discrete packets of information that evolve through discrete amounts of time.

\section{Normalizing}
Because the probabilistic interpretation of both statevectors and wavefunctions, these two objects have to be normalize upon measurement, thus making the total sum of probabilities $1$. How to normalize these objects is a perfect example to illustrate the difference in complexity that I see between quantum mechanics and quantum computation. 

 For a wavefunction $\Psi$ that represents a particle, the probability of finding the particle in a point $x$ is $\abs{\Psi(x,t)}^2$. Then, the wavefunction has to be normalized like the following:
\begin{equation}
\int_{-\infty}^{+\infty}\abs{\Psi(x,t)}^2\, dx\ = 1
\label{eq:norm_wave}
\end{equation}

Since the wavefunctions evolves over time with the Schrödinger equation, see Fig.\ref{fig:schro}, any solution this equation has to be normalized. Luckily, if the wavefunction is normalized at time $t=0$ it stays this way, in other words, the Schrödinger equations preserves the normalization of the wavefunction \cite{IntroQM:normalizing}. 

\begin{figure}
	$$
	i\hslash \pdv{\Psi}{t} = -\frac{\hslash^2}{2m}\pdv[2]{\Psi}{x}+V\Psi
	$$
	\caption{\textbf{Schrödinger's Equation}. Where $\hslash$ is $h/2\pi$, and $V$ is the potential energy function.}
	\label{fig:schro}
\end{figure}

We can prove that the equation, preserves \ref{eq:norm_wave}, starting from the trivial equality:
$$
\dv{t} \int_{-\infty}^{+\infty}\abs{\Psi(x,t)}^2\, dx\ = \pdv{t} \int_{-\infty}^{+\infty}\abs{\Psi(x,t)}^2\, dx
$$
By the product rule we have that\footnote{A partir d'ara escriuré  $\Psi(x,t)$ simplement com $\Psi$ per no fer les equacions tan enrevessades.}:
$$
\pdv{t}\abs{\Psi}^2 = \pdv{t}(\Psi\Psi^*) = \Psi^*\pdv{\Psi}{t} + \pdv{\Psi^*}{t}\Psi
$$
Now the Schrödinger equation says that 
$$
\pdv{\Psi}{t} = \frac{i\hslash}{2m}\pdv[2]{\Psi}{x} -  \frac{i}{\hslash}V\Psi
$$
then by taking the complex conjugate we have that 
$$
\pdv{\Psi^*}{t} = - \frac{i\hslash}{2m}\pdv[2]{\Psi^*}{x} + \frac{i}{\hslash}V\Psi^*
$$
so
$$
\pdv{t}\abs{\Psi}^2 = \frac{i\hslash}{2m} \left( \Psi^*\pdv[2]{\Psi}{x} - \pdv[2]{\Psi^*}{x}\Psi \right) = \pdv{x} \left[ \frac{i\hslash}{2m} \left(\Psi^*\pdv{\Psi}{x} - \pdv{\Psi^*}{x}\Psi\right)\right]
$$
finally we can evaluate the integral from the start:
$$
\dv{t} \int_{-\infty}^{+\infty}\abs{\Psi(x,t)}^2\, dx\ = \frac{i\hslash}{2m} \left( \Psi^*\pdv{\Psi}{x} - \pdv{\Psi^*}{x}\Psi \right) \Big|_{-\infty}^{+\infty}
$$
Because $\Psi(x,t)$ has to go to zero when x goes to either infinity, is true that:
$$
\dv{t} \int_{-\infty}^{+\infty}\abs{\Psi(x,t)}^2\, dx\ = 0
$$
Thus the integral is constant and went $\Psi$ is normalized at $t=0$, it stays that way for any $t$ (positive of course). \todo{falta la prueba para la normalización de un statevector}

\chapter{Polarització d'un fotó}
\label{appendix:optics}
En l'equació \ref{eq:photon_state} he exclusit el concepte de fase, que determina el tipus de polarització que té un fotó. Hi han 3 tipus:
\begin{enumerate}
	\item \textbf{Linear:}
	Un fotó té polarització lineal quan els angles de la fase $\alpha_x$, $\alpha_y$ en els estats base $\ket{x},\ket{y}$ són iguals:
	\begin{align*}
		\ket{\nearrow} &= \cos(\theta) e^{i\alpha_x}\ket{x} + \cos(\theta)e^{i\alpha_y}\ket{y} \\
		&= [\cos(\theta)\ket{x} + \sin(\theta)\ket{y}]e^{i\alpha}
	\end{align*}
	On $\alpha=\alpha_x=\alpha_y$.
	\item \textbf{Circular:}
	Quan els angles $\alpha_x$, $\alpha_y$ son separats per exactament $\frac{\pi}{2}$ i la amplitud per les dos bases és la mateixa:
	\begin{align*}
		\ket{\nearrow} &= \frac{1}{\sqrt{2}}\cos(\theta)e^{i\alpha_x}\ket{x} \pm i\frac{1}{\sqrt{2}}\sin(\theta)e^{i\alpha_y}\ket{y} \\
					   &= [\cos(\theta)e^{i\alpha_x}\ket{x} \pm i\sin(\theta)e^{i\alpha_y}\ket{y}]\frac{1}{\sqrt{2}}
	\end{align*}
	On el signe $\pm$ indica la diferencia entre la diferencia entre la polarització circular cap a la dreta o la esquerra, amb $+$ i $-$, respectivament.
	\item \textbf{El·líptica:}
	On els angles $\alpha_x$, $\alpha_y$ son diferents per una quantitat arbitraria\footnote{Però que no sigui la quantitat que dona a terme la polarització circular.}:
	$$
	\ket{\nearrow} = \cos(\theta)e^{i\alpha_x}\ket{x} + \sin(\theta)e^{i\alpha_y}\ket{y}
	$$
	Aquest és el cas més general.
	
\end{enumerate}

\chapter{Complexitat i algoritmes quàntics}
\label{complexity}
En ciència de la computació existeix el concepte de \textit{Big-O Notation}, una forma d'expressar lo eficients que són els algoritmes per fer certes tasques, en altres paraules la complexitat dels algoritmes. Bàsicament es una forma de classificar-los segon la rapidesa que tenen en ver la tasca que els correspon, aquesta rapidesa no és mesura en segons, degut a que aquesta mètrica pot variar d'ordinador a ordinador per les diferencies en hardware que aquest poden tindre. En canvi es mesure en nombre d'operacions o temps directament, però sense unitats. 

La \textit{Big-O Notation} consisteixes en definir el temps màxim que necessita un algoritme, es denota com $O(\cdot)$ on l'argument usualment depèn de $n$ que és la mida del input al algoritme, per exemple un algoritme de cerca ha de cercar a través de $n$ coses. Com a un exemple més concret tenim que un l'algoritme de cerca de cadenes binaries corre en un temps $O(\log_{2} n)$, on $n$ és el nombre de cadenes entre les quals ha de cercar. Recorda que la notació $O(\cdot)$ és el màxim, es a dir es \textit{upper bounded}, això significa que $\log_{2} n$ és la quantitat de temps més gran en la que es troba la cadena, també es possible que es trobi-s'hi a la primera comprovació que es va\footnote{Que la primera cadena que es cerca, és la que s'ha de trobar.}, llavors l'algoritme acabaria en un temps $O(1)$. Simplement és una manera de mirar lo eficients que són els algoritmes en relació a la mida del input que tenen.

Amb aquesta notació tenim una manera de comparar la eficiència que tenen els algoritmes quàntics amb la del clàssics que tenen la mateixa funció. 

\section{Algoritme de Grover}
Al 1996, Lov Grover va presentar un algoritme quàntic per cercar en dades desordenades \cite{Grover_96} (e.g. cercar el número de telèfon d'una persona en una llista desordenada). Per aquest problema un algoritme clàssic té una complexitat de $O(N)$ cerques\footnote{Una cerca és quan es verifica si un element de la llista és l'element que es cerca.}, mentre que l'algoritme de Grover té una complexitat de $O(\sqrt{N})$, sent substancialment més eficient. En les paraules de Grover \cite{Grover_96} (adaptades), un ordinador clàssic per tindre un probabilitat de $\frac{1}{2}$ de trobar el número de telèfon d'una persona en una llista desordenada necessita mirar a un mínim de $\frac{N}{2}$ números, mentre que amb el seu s'obté el número de telèfon en només $O(\sqrt{N})$ passos\footnote{Per passos entenc que es refereix al nombre de vegades que es mira al oracle, es a dir el nombre que de vegades que es verifica si s'ha trobar el que es cerca.}.

L'algoritme funciona de la següent manera: 

\chapter{Codi}
En l'apèndix actual presentaré el codi que he utilitzat al llarg del treball. Està organitzat segons el moment en el qual he referenciat el codi en el text.

\section{Part I}
\subsection{Capítol 3}

\paragraph{Regressió lienal}
\label{lst:linear_regression}
Codi per efectuar una regressió lineal a dades que es generen al atzar en el mateix arxiu, l'utilitzo per poder generar una gràfic per il·lustrar un exemple de regressió lienal. Aquest troç de codi l'he tret d'un repositori de GitHub\footnote{text}

\begin{lstlisting}[language=Python, caption=Regressió lineal]
	import numpy as np
	from matplotlib import pyplot as plt
	import matplotlib
	
	font = {'family' : 'Helvetica',
		'size'   : 18}
	
	matplotlib.rc('font', **font)
	
	
	# generate the data
	np.random.seed(222)
	X = np.random.normal(0,1, (200, 1))
	w_target = np.random.normal(0,1, (1,1))
	# data + white noise
	y = X@w_target + np.random.normal(0, 1, (200,1))
	
	
	# least squares
	w_estimate = np.linalg.inv(X.T@X)@X.T@y
	y_estimate = X@w_estimate
	
	
	# plot the data
	plt.figure(figsize=(15,10))
	plt.scatter(X.flat, y_estimate.flat, label="Prediccio")
	plt.scatter(X.flat, y.flat, color='red', alpha=0.4, label="Dades")
	plt.tight_layout()
	plt.title("Regressio per diferencia de quadrats")
	plt.legend()
	plt.savefig("least_squares.png")
	plt.show()
\end{lstlisting}

